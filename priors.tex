\documentclass{article}
\usepackage{amsmath}
\usepackage{amsfonts}
\usepackage[margin=1in]{geometry}
\title{Setting priors based using historical data}
\author{Alex Yakubovich}

\begin{document}
\maketitle
\section{Beta-Binomial Model}\label{sec:bb-model}
A beta prior can be parameterized in terms of an expected conversion rate $m$ and a shape parameter $\beta$. The former can be estimated from historical data.  The mode of the beta distribution is 
$$m = \frac{\alpha - 1}{\alpha + \beta -2}$$
If $\alpha$ is fixed, we can easily find a value of $\beta$ that satisfies this equation:
\begin{align}
\alpha - 1 &= m(\alpha + \beta - 2) \\
&= m\alpha + m\beta - 2m
\end{align}

Solving for $\beta$:
\begin{align}
\beta &= \frac{\alpha - 1 - m\alpha + 2m}{m} \\
&= 2 - \alpha + \frac{1}{m}\left(\alpha-1\right) \label{eq:beta}
\end{align}

Thus, the prior is specified by first learning the expected conversion rate from historical data, and then mapping it to $\beta$ using equation \ref{eq:beta}. $\alpha$ can be chosen to give the desired variance of the, which can be determined graphically. The variance of the prior increases with $\alpha$. 

\section{Zero-Inflated Log-normal model}

The conversion component of this model is just another beta-binomial model, so we can specify $(\alpha, \beta)$ as per section \ref{sec:bb-model}. The second component of the model describes the revenue amongst spenders. %Recall the model:
%\subsection{Prior}
The prior distribution is specified as a product of a conditional and marginal distribution:
\begin{align}
\sigma^2 &\sim \text{Inv}\chi^2 (\nu_0, \sigma^2_0) \Leftrightarrow \text{Inv-Gamma}\left(\frac{\nu_0}{2}, \frac{\nu_0 \sigma^2_0}{2} \right) \label{eq:inv-gamma-prior}\\
\mu | \sigma^2, \text{data} &\sim \mathcal{N} \left(\mu_0, \frac{\sigma^2}{\kappa_0} \right)
\end{align}

%\subsection{Posterior}
%\begin{align}
%\sigma^2 | \log(\text{data}) &\sim \text{Inv}\chi^2 (\nu_n, \sigma^2_n) \\
%\mu | \log(\text{data}), \sigma^2 &\sim \mathcal{N} \left(\mu_n, \frac{\sigma^2}{\kappa_n}\right)
%\end{align}

The prior is specified is defined over $\sigma^2, \mu | \sigma^2$.  We can summarize the prior over $\sigma^2$ by it's mode. The mode of the inverse gamma prior (Equation \ref{eq:inv-gamma-prior}) is given by:
\begin{align}
\sigma^2_m &= \frac{\beta}{\alpha + 1}, \text{ where } \alpha=\nu_0/2, \beta=\nu_0 \sigma^2_0/2 \\
&= \frac{\nu_0 \sigma^2}{\nu_0/2 + 1} \\
&= \frac{\nu_0 \sigma^2_0}{\nu_0 + 2}
\end{align}

Next, set the expected value  of the prior to the average revenue amongst spenders from historical data, call it $\mu_{obs}$:
\begin{align}
\log \mu_{obs} &= \mathbb{E} [\mu | \sigma^2, \log(\text{data})] \\
&= \mu_0 + \frac{\sigma^2_m}{2} \\
\mu_0 &= \log \mu_{obs} - \frac{\sigma^2_m}{2} \\
&= \log{\mu_{obs}} - \frac{1}{2} \left[\frac{\nu_0 \sigma^2_0}{\nu_0 + 2} \right] \label{eq:mu0-hyperparameter}
\end{align}
Thus, using equation \ref{eq:mu0-hyperparameter} we can parameterize the hyperparameter $\mu_0$ using the historical revenue amongst spenders $\mu_{obs}$.
\end{document}
